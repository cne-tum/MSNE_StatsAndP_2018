


\documentclass[10pt, handout]{beamer}
\setbeamertemplate{navigation symbols}{}
\usefonttheme{serif} 
\usepackage{amsmath}
\usepackage{amssymb}
\usepackage{graphicx}
\usepackage{cite}
\usepackage{color} 
\usepackage{setspace}
\usepackage{hyperref}

\newcommand{\xx}{{\bf{x}}}

\begin{document}
\title{Machine Learning I Lecture II:\\ Tutorial on probability theory}   
\author{Jakob H Macke\\ Max Planck Institute for Biological Cybernetics\\ Bernstein Center for Computational Neuroscience} 
\date{XY.XY.2012} 

\frame{\titlepage} 

%\frame{\frametitle{Today: Back to basics of probability theory}} 


\frame{\frametitle{Plan for today}\tableofcontents} 


\section{Discrete probability distributions} 

\frame[shrink=10]{\frametitle{To specify a discrete random variable, we need a sample space and a probability mass function.}

\begin{itemize}
\item \alert{Sample space $\Omega$}: Possible 'states' $x$ of the random variable $X$ \\(outcomes of the experiment, output of the system, measurement). Examples: [on board]
\item \pause Discrete random variables either have a finite or countable number of states.
\item \pause \alert{Events:} Possible combinations of states ('subsets of $\Omega$')
\item \pause \alert{Probability mass function $P(X=x)$}: A function which tells us how likely each possible outcome is.
%\item %Rules: 
\begin{align}
P(X=x)&=P_X(x)=P(x)\\
P(x)&\geq 0 \mbox{ for each } x\\
\sum_{x \in \Omega} P(x)&=1\\
P(A) = P(x \in A) &=\sum_{x \in A} P(X=x)\\
\end{align}
\item We write: $X|q \sim \mbox{Binomial}(q)$
\item Bernoulli, Binomial, Multinonomial, Poisson:  [on board]
\end{itemize}

} 
\frame{\frametitle{Conditional probability: Updating probabilities after we obtain information.}
\begin{itemize}
\item \alert{Conditional probability:} 'Recalculated probability of event A after someone tells you that event A happened.' 
\begin{align}
P(A|B)&= \frac{P(A \cap B)}{P(B)}\\
\pause P(A \cap B)&= P(A|B) P(B)
\end{align}
\item \pause Examples: Rolls of a die
 [on board]
\item \pause Bayes Rule: 
\begin{align}
P(B|A) = \frac{P(A|B) P(B)}{P(A)}
\end{align}
\end{itemize}

}

\frame{\frametitle{Expectation and  variance characterize the mean value of a random variable and its dispersion.}
\begin{itemize}
\item Expectation (or mean): $E(X)= \sum_{x} P(X=x) x$ 
\item \pause Expectation of a function:  $E(f(X))= \sum_{x} P(X=x) f(x)$ 
\item \pause Moments= expectation of power of $X$: $M_k= E(X^k)$
\item \pause Variance: Average (squared) fluctuation from the mean
\begin{align}
 \mbox{Var}(X)&= E((X-E(X))^2)\\
 &= E(X^2)- E(X)^2\\
 &= M_2-M_1^2
\end{align}
\item \pause Standard devation: Square root of variance.
\item Illustration and examples: [on board]
\pause Aside: Difference between expectation/variance of random variable and empirical average/variance.
\end{itemize}

}


\section{Multivariate distributions: Joints, conditionals and marginals} 
\frame{\frametitle{Bivariate distributions characterize systems with two observables.}
\begin{itemize} 
\item Example [on board]     
\item \alert{Joint distribution:} $P(X=x, Y=y)$, a list of all probabilities of all possible pairs of observations
\item \pause \alert{Marginal distribution:} $P(X=x)=\sum_y P(X=x, Y=y) $
\item \pause \alert{Conditional distribution:} $P(X=x|Y=y) = \frac{P(X=x, Y=y)}{P(y=y)}$
\item \pause $X |Y$ has distribution $P(X|Y)$, where $PX(|Y)$ specifies a 'lookup-table' of all possible $P(X=x| Y=y)$
\end{itemize}
\pause
\bf Conditioning and marginalization come up in Bayesian inference ALL the time: 'Condition on what you observe. Marginalize out the uncertainty'.


}

\frame{\frametitle{The importance of conditional probabilities: Interpreting medical tests}
%P(X=aids)=.001
%P(test positive|aids)=1
%P(test positive|heality)=.001
%P(aids|positive test)= 0.091
%Sensitivity: P(test positiv |disease)
%Specificity: P(test neagative | no disease)
%100 000 People

\begin{tabular}{l|ll|l}
~ & Positive Test & Negative Test & ~\\
\hline
HIV & 475 & 25 & 500\\
no HIV & 4975 & 94525 &  99500\\
\hline
~ & 5450 &  94550 & 100000
\end{tabular}

\vspace{1cm}
[on board]
\vspace{3cm}

\tiny Source: Statistical Methods for the Social Sciences, Agresti and Finaly, Prentice Hall-- not actual data
}

 \frame{\frametitle{Expectation and covariance of multivariate distributions:}
 \begin{itemize}
\item Conditional distributions are just distributions which have a (conditional) mean or variance. 
\item \pause Note: $E(X|Y)= f(Y)$. 'If I tell you what $Y$ is, what is the average value of $X$?.
%\item $E(X,Y)= \sum_{x,y} P(X=x, Y=y) (x,y)= (E(X), E(Y))$ 
\item \pause Covariance is the expected value of the product of fluctuations: 
\begin{align}
 \mbox{Cov}(X,Y)&= E\left((X-E(X) )(Y-E(Y) )\right)\\
 &= E(XY)- E(X)E(Y)\\
 \mbox{Var(X)}&= \mbox{Cov}(X,X)
\end{align}
 \end{itemize}
\pause Aside: One common way to construct bivariate random variables is to have a random variable whos parameter is another random variable. 
 }



 \frame{\frametitle{Independence of random variables}
 \begin{itemize}
 \item Intuitively, two \alert{events are independent} if knowing that the first took places tells us nothing about the probability of the second:  $P(A|B)= P(A)$
 \item  \pause $P(A) P(B)= P(A \cap B)$
 \item \pause Two \alert{random variables} are independent if the joint p.m.f. is the product of the marginals: $P(X=x,Y=y)=P(X=x) P(Y=y)$. 
 \item If $X$ and $Y$ are independent, we write $X \perp Y$. Knowing the value of $X$ does not tell us anything about $Y$.
 \item \pause If $X$ and $Y$ are independent, $\mbox{Cov}(X,Y)=0$.
 \end{itemize}
\pause Aside: Mutual information is a measure of how 'non-independent' two random variables are.
 }



\frame{\frametitle{Multivariate distributions are the same as bivariate distributions, just with more dimensions.}
\begin{itemize} 
\item $\mathbf{X},\xx$ are vector valued.
\item Mean: $E(\mathbf{X})= \sum_{\xx} \xx P(\xx)$
\item Covariance matrix: \begin{align}
\mbox{Cov}(X_i, X_j)&= E(X_iX_j)-E(X_i) E(X_j)\\
\mbox{Cov}(\mathbf{X})&= E(\mathbf{X}\mathbf{X}^\top)-E(\mathbf{X}) 
E(\mathbf{X})^\top 
 \end{align}
\item \pause Conditional and marginal distributions: Can define and calculate any (multi or single-dimensional) marginals or conditional distributions we need:  $P(X_1)$, $P(X_1, X_2)$, $P(X_1, X_2, X_3 |X_4)$, etc..
\end{itemize}

}
\section{} 
 
 




\section{Continuous probability distributions}

\frame{\frametitle{Continous random variables}
\begin{itemize}
\item A random variable $X$ is \alert{continuous} if its sample space $X$ is uncountable. 
\item In this case, $P(X=x)=0$ for each $x$.
\item \pause If $p_X(x)$ is a \alert{probability density function} for $X$, then 
\begin{align}
P(a < X <b) &=\int_a^b p(x) dx\\
P(a<X <a+dx) \approx p(a) \cdot dx
\end{align}
\item \pause The \alert{cumulative distribution function} is $F_X(x)=P(X<x)$. We have that $p_X(x)=F'(x)$, and $F(x)=\int_{-\infty}^x p(s) ds$.
\item  \pause 
More generally: If $A$ is an event, then
\begin{align}
P(A)&=P(X \in A) =\int_{x \in A}  p(x) dx\\
P(\Omega)&=P(X \in \Omega) =\int_{x \in \Omega} p(x) dx=1
\end{align}
\item Example: Uniform, Exponential, Beta  [on board]
\end{itemize}
}



\frame{\frametitle{People will often say probability when they mean probability density.}
\begin{itemize}
\item Probability density functions do not satisfy the definitions of probability (e.g. they can bigger than $1$). However, people (including your lecturer) will often be sloppy and write things like $P(X=x)$ and say 'the probability of $X$' when they really mean 'the probability density of $X$ evaluated at $x$'.
\item \pause Similarly, people (including your lecturer) will often be sloppy and write integrals or say 'we need to integrate our $X$' when they write down general formulas-- if these formulas are applied to discrete random variables, the integrals would need to be replaced by sums.
\item \pause This might be bad practice, but it is usually clear from the context whether a random variable is discrete or continuous. In addition, it is good preparation for reading papers---many machine learning papers are very sloppy about usage of these terms.
\end{itemize}
}



\frame{\frametitle{Mean, variance, and conditioning on events are the same as the discrete case, just with sums replaced by integrals.}
\begin{itemize}
\item Mean: $E(X)= \int_x x  \cdot p(x) dx$\\
\item Variance: $\mbox{Var}(X)= E(X^2)- E(X)^2$
\item Example: Uniform, Exponential [on board]
\item \pause If $X$ has pdf $p(x)$, then $X | (X \in A)$ has pdf 
\begin{align}
p_{X|A}(x)=\frac{p(x)}{P(A)}=\frac{p(x)}{\int_{x \in A} p(x) dx}
\end{align}
\item \pause Only makes sense if $P(A)>0$~!
\item Example: Uniform, Exponential [on board]
\end{itemize}
}


%\frame{\frametitle{The univariate Gaussian}

%[on board]

%}

\frame{\frametitle{Bivariate continuous distributions: Marginalization, Conditioning and Independence}
\begin{itemize}
\item $p_{X,Y}(x,y)$, joint probablity density function of $X$ and $Y$
\item $\int_x \int_y p(x,y)dx dy=1$
\item \pause \alert{Marginal distribution:} $p(x)= \int_{-\infty}^\infty p(x,y) dy$
\item \pause \alert{Conditional distribution:} $p(x|y)= \frac{p(x,y)}{p(y)}$ 
\item Note: $P(Y=y)=0$! Formally, conditional probability in the continuous case can be derived using infinitesimal events.
\item \alert{Independence:} $X$ and $Y$ are independent if $p_{X,Y}(x,y)=p_X(x)p_Y(y)$
\end{itemize}
}

%\section{The Gaussian distribution}
 

%\frame{\frametitle{The multivariate Gaussian}

%[on board]

%}
 

%\frame{\frametitle{The magic formula: Conditional distributions in the multivariate Gaussian}

%[on board]

%}
 
\end{document}



