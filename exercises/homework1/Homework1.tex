\documentclass{exam}

\usepackage{amsmath}

\usepackage{amssymb}

\usepackage{graphicx}


\usepackage{hyperref}

\pagestyle{headandfoot}
\runningheadrule
\firstpageheader{Machine Learning I}{Homework I}{October XY, 2012}

\title{Homework I: Gaussians and linear regression}
\date{}
\begin{document}
\maketitle
\thispagestyle{headandfoot}

\begin{center}
  {\fbox{\parbox{5.5in}{\centering
Solutions to this exercise sheet are to be handed in
 before the tutorial session on Friday, XY.XY.12 Code (.m-file) and output (.mat-file, figures as .pdf-files) for the  matlab-questions should be submitted by email to \texttt{XY@student.uni-tuebingen.de}. Use comments to explain your code. All code and plots should also be handed in as hardcopy.}}}
\end{center}
\vspace{.5cm}
\begin{questions}
\question[15]{\bf Mathematical preliminaries: Manipulating Gaussian densities} Consider the function $f(x)=\frac{1}{Z}\exp(ax^2+bx+c)$, with $a<0$, and the Gaussian density $g(x)=\frac{1}{\sqrt{2\pi\sigma^2}}\exp\left(-\frac{1}{2}\left(\frac{x-\mu}{\sigma}\right)^2 \right)$ 
\begin{parts}
\part Find the maximum of $\log(f(x))$ as well its curvature (i.e. second derivative) at the maximum. 
\part Show that, for appropriate choices of $a$, $b$, $c$, and $Z$, $f(x)$ is equivalent to $g(x)$.    [Hint: Multiply out the square term in the exponent of $g(x)$, and equate the coefficients of $f(x)$ and $g(x)$.] Express $\mu$  and $\sigma$ in terms of $a$, $b$, and $c$.
\end{parts}

\question[25] {{\bf Linear regression [MATLAB]}.  } Suppose you are given training-inputs $D=\{\mathbf{x_1}, \ldots , \mathbf{x_N}\}$, where each $\mathbf{x}_i$ is an $M\times 1$-dimensional vector, and outputs $\{y_1, \ldots,  y_N\}$  You want to fit a function of the form $f(\mathbf{x},\omega)= \omega^\top \mathbf{x} =\sum_{i=1}^M w_i x_i$ to this data by minimising the cost function 
\begin{align}
L(\omega,D)= \sum_{n=1}^N (f(\mathbf{x_n},\omega)-y_n)^2 + \lambda  \left(\omega^\top \omega\right)\nonumber
\end{align} 
\begin{parts}
\part By taking the derivatives of $L$ with respect to each $\omega_i$ and setting them to $0$, show that this cost function is minimised for $\omega_\lambda=\left(\sum_n \mathbf{x_n}\mathbf{x_n}^\top +\lambda \mathbf{I_M}\right)^{-1}\left( \sum_n \mathbf{x}_n y_n\right)$, where $\mathbf{I}$ is an identify matrix of size $M$.
\part  Load the variables in the file \texttt{Homework1.mat} and fit the parameters to the data \texttt{xTrain} and \texttt{yTrain} (where each row of \texttt{xTrain} corresponds to one data-point), using $\lambda=0$, i.e. no regularisation.  Plot the vector $\omega_o$ that you obtain.

\part Train multiple versions of your model on the training-set, using values $\lambda={1,5,10,25,50,75,100,250,500,750,1000}$. Make a plot that shows how the least-squares errors of your model on the training and the test set changes as a function of $\lambda$. [Note: Please make clearly labelled plots with meaningful axes.]
 \part Using this plot, decide which value of $\lambda$ you expect to give you the best generalization performance, and report the value. Use this model to predict the y-values for the data $\texttt{xValidation}$, save your predictions as a variable named $\texttt{yValidationPredicted}$ in a file \texttt{HomeWork1YourName.mat} and submit this file. 
\part Your friendly instructors included a column of 'ones' for the input data (i.e. the first entry of each $\mathbf{x}$ is $1$). Explain why (for this model), including a constant term in the input-data can be useful and results in a more flexibel regression model.
 \part $[$optional$]$ By inspecting the vector $\omega$ obtained for the best model, make a guess as to which of the dimensions of $x$ are important for predicting $y$, and which are irrelevant. 
\end{parts}

\end{questions}




\end{document}